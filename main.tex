%-------------------------------------------------------------------------------
%                      Template Naskah Skripsi
%               	Berdasarkan format JTETI FT UGM
%                   Modifikasi utk skripsi UTM
% 						(c) @gunturdputra 2014
% 						(c) @willnode 2021
%-------------------------------------------------------------------------------

%Template pembuatan naskah skripsi.
% PENTING: set mode "skripsi" atau "proposal"
\documentclass[proposal]{ftutmskripsi}

%Untuk prefiks pada daftar gambar dan tabel
\usepackage[titles]{tocloft}
\renewcommand\cftfigpresnum{Gambar\  }
\renewcommand\cfttabpresnum{Tabel\   }

%Untuk hyperlink dan table of content
\usepackage{hyperref}
\newlength{\mylenf}
\settowidth{\mylenf}{\cftfigpresnum}
\setlength{\cftfignumwidth}{\dimexpr\mylenf+2em}
\setlength{\cfttabnumwidth}{\dimexpr\mylenf+2em}

%Untuk Bold Face pada Keterangan Gambar
\usepackage[labelfont=bf]{caption}

%Untuk caption dan subcaption
\usepackage{caption}
\usepackage{subcaption}


%-----------------------------------------------------------------
%Disini awal masukan untuk data proposal skripsi
%-----------------------------------------------------------------
\titleind{TEMPLATE SKRIPSI UTM DENGAN MENGGUNAKAN \emph{TYPESETTING} \LaTeX}
\fullname{WILDAN M}
\idnum{18.04.1.1.1.00000}
\approvaldate{1 Januari 2021}
\degree{Teknik Informatika}
\yearsubmit{2021}
\program{Teknik Informatika}
\dept{Teknik}
\firstsupervisor{Sigit Basuki Wibowo, S.T., M.Eng.}
\firstnip{1976 0000 0000 00 1 001}
\secondsupervisor{Bimo Sunarfri Hantono, S.T., M.Eng.}
\secondnip{1976 0000 0000 00 1 002}

%-----------------------------------------------------------------
%Disini akhir masukan untuk data proposal skripsi
%-----------------------------------------------------------------

\begin{document}

\cover

\approvalpage


%-----------------------------------------------------------------
%Disini awal masukan Acknowledment
%-----------------------------------------------------------------
\acknowledgment
\begin{flushright}
\emph{Untuk Ibu, Bapak,\\dan Adik-adikku tercinta.}
\end{flushright}


\input{halaman/pengantar}

\tableofcontents
\addcontentsline{toc}{chapter}{DAFTAR ISI}
\listoftables
\addcontentsline{toc}{chapter}{DAFTAR TABEL}
\listoffigures
\addcontentsline{toc}{chapter}{DAFTAR GAMBAR}

%
%-----------------------------------------------------------------
%Daftar Singkatan [Optional]
%-----------------------------------------------------------------
\singkatan
\noindent

\begin{tabular}{p{20pt}p{3pt}l}
\textbf{A}\\
AJAX & & Asynchronous JavaScript and XML\\
AP & & Access Point\\
API & & Application Programming Interface\\
\\
\end{tabular}

\begin{tabular}{p{20pt}p{3pt}l}
\textbf{C}\\
CLI & & Command Line Interface\\
\\
\end{tabular}

\begin{tabular}{p{20pt}p{3pt}l}
\textbf{C}\\
DFM & & Discovered Full Mesh\\
\\
\end{tabular}

\begin{tabular}{p{20pt}p{3pt}l}
\textbf{E}\\
ERD & & Entity Relationship Diagram\\
\\
\end{tabular}

\begin{tabular}{p{20pt}p{3pt}l}
\textbf{F}\\
FTDI & & Future Technology Devices International\\
FUSE & & Filesystem in Userspace\\
\\
\end{tabular}

\begin{tabular}{p{20pt}p{3pt}l}
\textbf{I}\\
IP & & Internet Protocol\\
\\
\end{tabular}

\begin{tabular}{p{20pt}p{3pt}l}
\textbf{J}\\
JTETI & & Jurusan Teknik Elektro dan Teknologi Informasi\\
\\
\end{tabular}

\begin{tabular}{p{20pt}p{3pt}l}
\textbf{L}\\
LAN & & Local Area Network\\
\\
\end{tabular}

\begin{tabular}{p{20pt}p{3pt}l}
\textbf{O}\\
OSI & & Open Systems Interconnection\\
\\
\end{tabular}

\begin{tabular}{p{20pt}p{3pt}l}
\textbf{R}\\
RF & & Radio Frequency\\
\\
\end{tabular}

\begin{tabular}{p{20pt}p{3pt}l}
\textbf{S}\\
SDLC & & Software Development Life Cycle\\
SFTP & & Secure Shell File Transfer Protocol\\
SSHFS & & Secure Shell Filesystem\\
\\
\end{tabular}

\begin{tabular}{p{20pt}p{3pt}l}
\textbf{U}\\
UGM & & Universitas Gadjah Mada\\
USB & & Universal Serial Bus\\
\\
\end{tabular}

\begin{tabular}{p{20pt}p{3pt}l}
\textbf{V}\\
VRS & & Virtual Routing Structure\\
\\
\end{tabular}

\begin{tabular}{p{20pt}p{3pt}l}
\textbf{W}\\
WAP & & Wireless Access Point\\
WIT & & Western Indonesian Time\\
WLAN & & Wireless Local Area Network\\
WSN & & Wireless Sensor Network\\
\end{tabular}

\input{halaman/abstrak}

%-----------------------------------------------------------------
%Disini awal masukan untuk Bab
%-----------------------------------------------------------------
\include{konten/bab1}

\include{konten/bab2}

\include{konten/bab3}

\include{konten/bab4}

\include{konten/bab5}

%-----------------------------------------------------------------
%Disini akhir masukan Bab
%-----------------------------------------------------------------


%-----------------------------------------------------------------
% Disini awal masukan untuk Daftar Pustaka
% - Daftar pustaka diambil dari file .bib yang ada pada folder ini
%   juga.
% - Untuk memudahkan dalam memanajemen dan menggenerate file .bib
%   gunakan reference manager seperti Mendeley, Zotero, EndNote,
%   dll.
%-----------------------------------------------------------------
\bibliography{IEEEabrv,daftar-pustaka}
\addcontentsline{toc}{chapter}{DAFTAR PUSTAKA}
%-----------------------------------------------------------------
%Disini akhir masukan Daftar Pustaka
%-----------------------------------------------------------------

\end{document}